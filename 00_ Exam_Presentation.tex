\documentclass{beamer}  % beamer specific for presentations 
\usetheme{Boadilla}     % my theme 
\usepackage{graphicx}   % allows to include grapgics like 
\usepackage{subcaption} % to write subcations to images 
\usepackage{mathrsfs}   % to write maths formulas
\usepackage{listings}   % to write code
\usepackage{hyperref}   % to add external links 


% Colors personalization
\definecolor{myprimarycolor}{RGB}{69,108,81}    % Defining a custom primary color (dark salvia)
\definecolor{mysecondarycolor}{RGB}{50,150,50}  % Defining a custom secondary color (salvia) 
\setbeamercolor{structure}{fg=myprimarycolor}   % Setting the structure (e.g., text) color to the primary color
\setbeamercolor{block title}{fg=white,bg=mysecondarycolor} % Setting block title text color to white on background of the secondary color
\setbeamercolor{block body}{bg=white}           % Setting the block body background color to white


% Front page with title, institute, author, date and logo
\title{Analisi Incendi Luglio-Agosto 2023, Palermo, IT}   % Setting the presentation title
\institute{Alma Mater Studiorum - Università di Bologna\\ Esame Telerilevamento Geo-Ecologico}  % Setting the institute
\author{Giulian Capasso}  % Setting the author's name
\date{07/09/2023}         % Setting the date
% logo
\titlegraphic{
  \vspace{1cm}
  \centering
  \includegraphics[height=1.5cm]{alma-mater-studiorum-universita-di-bologna-logo-vector.png}
}

% Personalization of the bar at the bottom of the page (footline)
\setbeamertemplate{footline}{
  \leavevmode%
  \hbox{%
    \begin{beamercolorbox}[wd=.23\paperwidth,ht=2.5ex,dp=1.125ex,center]{author in head/foot}%
      \usebeamerfont{author in head/foot}\insertshortauthor
    \end{beamercolorbox}%
    \begin{beamercolorbox}[wd=.53\paperwidth,ht=2.5ex,dp=1.125ex,center]{title in head/foot}%
      \usebeamerfont{title in head/foot}\insertshorttitle
    \end{beamercolorbox}%
    \begin{beamercolorbox}[wd=.25\paperwidth,ht=2.5ex,dp=1.125ex,right]{date in head/foot}%
      \insertframenumber{} / \inserttotalframenumber\hspace*{1.3cm}
    \end{beamercolorbox}}%
  \vskip0pt%
}

% Code style pesonalization 
\lstdefinestyle{mystyle}{  
    basicstyle=\tiny\ttfamily,  % Setting the basic code style to a small and monospaced font
    breaklines=true,            % Enabling line breaks within code listings
}


% Here I begin my document 
\begin{document}
\begin{frame}
\titlepage
\end{frame}

% Summary with itemization 
\begin{frame}{Sommario}
    \begin{itemize}
        \item Inquadramento e scopi del progetto
        \item Metodologia
        \item Indagine Preliminare
        \item Analisi 
\begin{itemize}
            \item Importazione dati e Visualizzazione
            \item Indici Spettrali
            \item Land Cover Classification
            \item Land Surface Temperature
        \end{itemize}
    \item Conclusioni
    \end{itemize}
\end{frame}

% second frame divided in 2 columns and 4 spaces to add to parts of code and the 2 relative plots 
\begin{frame}{Inquadramento e scopi del Progetto}
\frametitle{Inquadramento e scopi del Progetto}
\vspace{0.5cm}
\begin{columns}
  \begin{column}{0.20\textwidth}
    \centering
    \includegraphics[width=\linewidth]{Exam/Inquadramento.png}
  \end{column}
  \begin{column}{0.34\textwidth}
    \centering
    \includegraphics[width=\linewidth]{Mappaincendicopernicus.png}
  \end{column}
  \begin{column}{0.39\textwidth}
    \centering
    \includegraphics[width=\linewidth]{Exam/Mondello.png}
  \end{column}
\end{columns}
\vspace{0.5cm}
\small % Cambia il font del testo a piccolo
Dal 24 Luglio al 10 Agosto, la periferia della città di Palermo è stata colpita da incendi che hanno causato oltre a evaquazioni e assenza di acqua ed elettricità, anche un'importante perdita di biodiversità. Il progetto punta a:
\vspace{0.2cm}
\begin{itemize}
    \item Comparare le immagini prima e dopo gli incendi
    \item Calcolare la perdita di vegetazione
    \item Visualizzare le alterazioni di temperatura del suolo 
\end{itemize}
\end{frame}

% third frame, itemization and images upload 
\begin{frame}{Metodologia}
    \begin{itemize}
        \item Utilizzo di dati satellitari da Sentinel-2B e Landsat 8-9 
        \item Elaborazione dati tramite il software R
        \item Presentazione in \LaTeX
    \end{itemize}
    \vspace{0.5cm}
    \centering
    \includegraphics[width=0.7\textwidth]{Exam/sentinel2b_bands.jpeg}
    {\tiny Sentinel-2B}
    \vspace{0.2cm}
    \includegraphics[width=0.7\textwidth]{Exam/Landsat bands.png}
    {\tiny Landsat 8}
\end{frame}

% Fourth frame about the prequel - just 2 images uploaded
\begin{frame}[fragile]{Indagine Preliminare}
 \framesubtitle{Analisi intera area dal 1990 al 2023}
 \begin{center} 
            \includegraphics[height=3cm]{Exam/Prequel_Tc.png}
       \end{center}
            \vspace{0.1cm}
             \begin{center} 
                \includegraphics[height=3cm]{Exam/Prequelnir.png}
\end{center}
     \end{frame}
 
% Fifth frame for classification in which again I use the comun division    
\begin{frame}[fragile]
    \frametitle{\scriptsize Classificazione terreno - Trend classi e Differenza tra NDVI}
 \begin{center} 
            \includegraphics[height=3.5cm]{Exam/prequeltrend.png}
            \end{center}
             \begin{center} 

             \vspace{0.5cm}
                \includegraphics[height=2.5cm]{Exam/prequelndvi.png}
\end{center}
     \end{frame}

% Sixth frame, 2 columns 2 rows > I sow some code here so I use the lstlisting function
\begin{frame}[fragile]{Analisi}
    \framesubtitle{Importazione dati e Visualizzazione - Focus sull'incendio - 2022 e 2023}
    \begin{columns}
        \begin{column}{0.6\textwidth}
            \begin{lstlisting}[style=mystyle]
#2022
rlist_22 <- list.files(pattern = "2022") 
import_22 <- lapply(rlist_22, raster)
tgr_22 <- stack(import_22)
ext <- c(330000, 371000, 4200220, 4237800)
crop_22 <- crop(tgr_22, ext) 
png("Truecolor2022.png")
plotRGB(crop_22, r=4, g=3, b=2, stretch ="lin")
dev.off()
            \end{lstlisting}
        \end{column}
 \begin{column}{0.3\textwidth}
            \includegraphics[width=\textwidth]{Exam/Truecolor2022.png}
        \end{column}
    \end{columns}
 \begin{columns}
        \begin{column}{0.6\textwidth}
 \begin{lstlisting}[style=mystyle]
#2023
rlist_23 <- list.files(pattern = "2023") 
import_23 <- lapply(rlist_23, raster)
tgr_23 <- stack(import_23)
crop_23 <- crop(tgr_23, ext) 
plotRGB(crop_23, r=4, g=3, b=2, stretch="lin")
png("Truecolor2023.png")
plotRGB(crop_23, r=4, g=3, b=2, stretch ="lin")
dev.off()
 \end{lstlisting}
\end{column}
        \begin{column}{0.3\textwidth}
            \includegraphics[width=\textwidth]{Exam/Truecolor2023.png}
        \end{column}
    \end{columns}

\end{frame}

% Seventh frame, I add an image to show boht the natural colors plot and the False colors With NIR on RED
\begin{frame}[fragile]
    \frametitle{\scriptsize True Colors - False Colors ggplot2 e patchwork}
              \includegraphics[height=7.5cm]{Exam/ggrgb.pdf}
                \centering 
\end{frame}

% Eighth Frame, NDVI and NBR, with subtitle and image centered with a differenc fuciotn because it did bug
\begin{frame}[fragile]{Indici Spettrali}
    \framesubtitle{NDVI - NBR}
 \begin{itemize}
        \item \scriptsize NDVI (Normalized Difference Vegetation Index): (NIR-RED)/(NIR+RED)
        \item NBR (Normalized Burn Ratio): (NIR-SWIR)/(NIR+SWIR)
\end{itemize}
            \begin{center} 
        \includegraphics[height=6.5cm]{Exam/ndvi+nbr.pdf}
    \end{center}
\end{frame}

% Ninth frame with subtitle in which I show other Incices like the BAI and SAVI and some script to describe the formula 
\begin{frame}[fragile]
\frametitle{\scriptsize Altri Indici: BAI(Burned Area Index); SAVI(Soil Adjusted Vegetation Index)}
           \begin{lstlisting}[style=mystyle]
# BAI=(SWIR-NIR)/(SWIR+NIR)+A)*(1+A)*(1+B)
# SAVI=(NIR-RED)/(NIR+RED+L*(1+L))||L=0.5
            \end{lstlisting}
       \begin{center}
\includegraphics[height=7.5cm]{Exam/bai+savi.pdf}  
       \end{center}
\end{frame}

% Tenth frame with subtitle, showing two images with two different palettes and showing the overall comparisons of the indices
% but now showed as divverece meaning 2022 minus 2023 NDVI, NBR, BAI and SAVI
\begin{frame}[fragile]
\frametitle{\scriptsize Confronto Indici su Differenze tra il 2022 e 2023 :: Pre-Post incendi}
\framesubtitle{\tiny Palette viridis(inferno) 2. Palette personalizzata ad alto contrasto}
\includegraphics[height= 5 cm]{Exam/tot.png} 
\end{frame}

% Eleventh frame columns divisions ti show true colors image and classified image on de right 
% and code to do so wuth percentages on the left 
\begin{frame}[fragile]{Land Cover Classification}
    \framesubtitle{Assegnazione di classi e mappatura di vegetazione}
    \begin{columns}
        \begin{column}{0.45\textwidth}
            \begin{lstlisting}[style=mystyle]
#2022
set.seed(42) 
clas22<-unsuperClass(crop_22,nClasses=5)
par(mfrow=c(1,2))
plot(clas22$map,col=clr)+title(main="Soil Classification 2022")
plotRGB(crop_22,r=4,g=3,b=2,stretch="lin")

freq(clas22$map)
# 1 - 237362 - cities / bare rock  
# 2 -1048263 - dry / burned land
# 3 -1354468 - water bodies
# 4 - 615922 - vegetation
# 5 - 593133 - vegetation

# Percentage of classes in 2022
tot_land_22<-237362+1048263+615922+593133
tot_veg_22<-615922+593133
percent_veg_22<-tot_veg_22*100/tot_land_22 #48.46533 %VEG

percent_cities_rocks_22<-237362*100/tot_land_22 #9.514727 %C/R

tot_dry_22<-1048263 
percent_dry_22<-tot_dry_22*100/tot_land_22 #42.01994 %DRY LAND
            \end{lstlisting}
        \end{column}
        \begin{column}{0.47\textwidth}
            \includegraphics[width=\textwidth]{Exam/Class22plot.png}
        \end{column}
    \end{columns}
\end{frame}

% 12th frame columns divisions ti show true colors image and classified image on de right 
% and code to do so wuth percentages on the left 
\begin{frame}[fragile]
\frametitle{\scriptsize Assegnazione di classi e mappatura di vegetazione}
    \begin{columns}
        \begin{column}{0.45\textwidth}
            \begin{lstlisting}[style=mystyle]
#2023
set.seed(42) 
clas23<-unsuperClass(crop_23,nClasses=5)
par(mfrow=c(1,2))
plot(clas23$map, col=clr)+title(main="Soil Classification 2023")
plotRGB(crop_23,r=4,g=3,b=2,stretch="lin")

freq(clas23$map)
# 1- 446911 - burned land
# 2- 946192 - dry land
# 3-1354606 - water bodies
# 4- 351882 - cities / bare rock
# 5- 752359 - vegetation 

# Percentage of classes in 2023
tot_land_23<-446911+946192+351882+752359
tot_veg_23<-752359 

percent_veg_23<-tot_veg_23*100/tot_land_23 #30.12637 %VEG

percent_cities_rocks_23<-351882*100/tot_land_23 #14.09025 %C/R

tot_dry_23<-946192+446911
percent_dry_23<-tot_dry_23*100/tot_land_23 #55.78338 DRY/BURNED LAND

            \end{lstlisting}
        \end{column}
        
        \begin{column}{0.47\textwidth}
            \includegraphics[width=\textwidth]{Exam/Class23plot.png}
        \end{column}
    \end{columns}

\end{frame}

% 13th frame with the code to create a dateframe with previous percentages and classes and then plot an istogram
\begin{frame}[fragile]
\frametitle{\scriptsize Rappresentazione Classi}
            \begin{lstlisting}[style=mystyle]
# Assegno Classi e percentuali
cl_22<-c("Cities/Rocks","Vegetation","Dry/Burned Land")
perc22<-c(9.5,48.5,42.0) #[...2023]
df0<-data.frame(cl_22,perc22) #[...2023]

# Graphs
is1<-ggplot(df0,aes(x=cl_22,y=perc22,fill=cl_22)) + geom_bar(stat="identity") + ggtitle("Land Cover percentage in 2022")labs(fill="Classes",x="Classes",y="%") + geom_text(aes(label=perc22),vjust=-0.3, size=3.5) + scale_fill_manual(values=c("honeydew3", "indianred3", "darkolivegreen")) + ylim(0,100)+theme_minimal()
#[... lo stesso per il 2023]

            \end{lstlisting}
        \vspace{0.5cm}
            \includegraphics[width=\textwidth]{Exam/Landcoveris.png}

\end{frame}

% 14th frame in which i show band 10 od landsat 8-9 and pair it with False color images to compare the areas of high temperature 
% and vegetation cover / burned ares 
% I also added the script but it was redundat to me
\begin{frame}[fragile]{Land Surface Temperature}
    \framesubtitle{Band 10 Thermal Infrared Landsat 8-9}
%  \begin{lstlisting}[style=mystyle]
%lst_22<-%raster("LC08_L2SP_189034_20220908_20220914_02_T1_ST_B10.TIF")
%lst_23<-%raster("LC09_L2SP_189034_20230818_20230822_02_T1_ST_B10.TIF")
%crop_t_22<-crop(lst_22,ext)
%crop_t_23<-crop(lst_23,ext)
%min_value_t2<-%min(min(values(crop_t_22)),min(values(crop_t_23)))
%max_value_t2<-%max(max(values(crop_t_22)),max(values(crop_t_23)))
%#plot 
%plot(crop_t_22,col=rev(heat.colors(100)),zlim=c(min_value_t2,max_value_t2),axes=FALSE, main = "2022")
%plot(crop_t_23,col=rev(heat.colors(100)),zlim=c(min_value_t2, max_value_t2),axes=FALSE, main = "2023")    
%            \end{lstlisting}
            \centering
\includegraphics[height=7.5cm]{Exam/confronto_t_nir.png}   
\vspace{0.5cm}                
\end{frame}

% This was a whole experiment that didnt work out to transforme reflectance brightness to Celsius
% I added it but then it was very hard to comment and discuss
% So I kept it just i case I need to show someone or I will be able to fix it 
% Bu i wont include it to the presentation
%\begin{frame}[fragile]
%\frametitle{\scriptsize Tentativo conversione B10 10 Landsat 8 in valori di temperatura}
%           \begin{lstlisting}[style=mystyle]
%# Coefficienti di calibrazione radiometrica per la banda 10
%RADIANCE_MULT_BAND_10<-3.8000E-04
%RADIANCE_ADD_BAND_10 <- 0.10000

%# Calcolo della radianza monocromatica (Lλ)
%radiance_band <- (crop_t_22 * RADIANCE_MULT_BAND_10) + RADIANCE_ADD_BAND_10

%# Costanti termodinamiche
%PLANCK_CONSTANT<-6.62607004e-34
%SPEED_OF_LIGHT<-299792458
%BOLTZMANN_CONSTANT<-1.38064852e-23

%# Lunghezza d'onda della banda termica 10 in metri
%WAVELENGTH_BAND_10<-10.8951e-6

%# Calcolo della temperatura radiometrica (TIR)
%temperature_radiometric<-(PLANCK_CONSTANT*SPEED_OF_LIGHT)/(WAVELENGTH_BAND_10 %* BOLTZMANN_CONSTANT)/log((radiance_band/0.99)+1)

%# Coefficienti di calibrazione per la temperatura radiometrica
%K1_CONSTANT_BAND_10<-774.8853
%K2_CONSTANT_BAND_10<-1321.0789

%# Calibrazione della temperatura radiometrica
%temperature_calibrated<-%K2_CONSTANT_BAND_10/log((K1_CONSTANT_BAND_10/temperature_radiometric)+1)

%# Visualizzazione della temperatura della superficie terrestre (LST)
%plot(temperature_calibrated,main = "Land Surface Temperature(LST)in Celsius")  %       \end{lstlisting}
%\href{https://https://supportsi.hexagon.com/s/article/Converting-Landsat-8-Thermal-Band-10-to-Temperature-values?language=en_US}
%            \centering
%\includegraphics[height=4.5cm]{Exam/temp.png}   
                  
%\end{frame}

%\begin{frame}[fragile]{Tentativo conversione in Celsius}
%    \framesubtitle{Google Earth Engine}      
%\end{frame}

% Conclusions with itemize 
\begin{frame}{Conclusioni}
    \centering
    \small
    \begin{itemize}
    \item  Grazie al telerilevamento è stato possibile individuare chiaramente le differenze nella copertura vegetale nelle aree colpite dagli incendi
            \vspace{0.2cm}
    \item NBR e BAI risultano gli indici che restituiscono più accuratezza
                \vspace{0.2cm}
    \item Dal 2022 al 2023 la vegetazione è diminuita dal 48.5\% al 30.1\%., mentre le aree spoglie sono aumentate dal 42.0\% al 55.7\%. La land cover analysis riflette chiaramente l'impatto degli incendi
                \vspace{0.2cm}
    \item I dati termici da Landsat mostra le aree bruciate con temperature più elevate rispetto alle altre
                \vspace{0.2cm}
    \item Future analisi 
    \begin{itemize}
        \item Monitoraggio della Ripresa Vegetativa
         \item Monotoraggio Incendi Dolosi
    \end{itemize}
    \end{itemize}
\end{frame}

% Bibliography and external liks 
\begin{frame}{Fonti e collegamenti esterni }
\frametitle{}
\begin{itemize}
\small 
\item \href{https://www.earthdatascience.org/courses/earth-analytics/multispectral-remote-sensing-modis/normalized-burn-index-dNBR/}{\scriptsize Wasser, L. and Cattau, Me. (2017) Work with the difference normalized burn index (NBR) - using spectral remote sensing to understand the impacts of fire on the landscape}
\item \href{https://www.mdpi.com/2504-3900/2/7/364}{\scriptsize Filipponi, F. (2018) Bais2: Burned area index for sentinel-2}
 \item \href{https://supportsi.hexagon.com/s/article/Converting-Landsat-8-Thermal-Band-10-to-Temperature-values?language=en_US}{\scriptsize Converting Landsat 8 Thermal Band 10 to Temperature values (2022) Hexagon help center}
\item \href{https://custom-scripts.sentinel-hub.com/custom-scripts/sentinel-2/savi/}{\scriptsize Sentinel-Hub(nd) Savi (soil adjusted vegetation index), Sentinel Hub custom scripts}
\item \href{https://scihub.copernicus.eu/dhus/#/home}{\scriptsize Scihub copernicus - Open access hub to download data}
\item \href{https://github.com/giulian-capasso/telerilevamento}{ \scriptsize giulian-capasso - Github account - repository telerilevamento}
\end{itemize}

\vspace{1.5cm}
\begin{center}
\scriptsize Grazie! 
Domande? 
\end{center}
\end{frame}



\end{document}




        
     
        
